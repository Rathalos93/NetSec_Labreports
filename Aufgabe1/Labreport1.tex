\documentclass[12pt]{article}


\usepackage{amssymb}
\usepackage{amsmath}
\usepackage[utf8]{inputenc}
\usepackage{lineno}
\usepackage{listings}
\usepackage[T1]{fontenc}
\usepackage[utf8]{inputenc}
\usepackage{lmodern}
\usepackage{eurosym}
\usepackage{listings}
\usepackage{microtype}
\usepackage{units}
\usepackage{color}
\usepackage{xcolor}
\usepackage{graphicx}
\usepackage{subfigure}
\usepackage{import}
\usepackage{url}
\usepackage{amsthm}
\theoremstyle{plain}

\lstset
{ %
  language=R,                     % the language of the code
  basicstyle=\footnotesize\ttfamily,       % the size of the fonts that are used for the code
  numbers=left,                   % where to put the line-numbers
  numberstyle=\tiny\color{gray},  % the style that is used for the line-numbers
  stepnumber=1,                   % the step between two line-numbers. If it's 1, each line
                                  % will be numbered
  numbersep=5pt,                  % how far the line-numbers are from the code
  backgroundcolor=\color{white},  % choose the background color. You must add \usepackage{color}
  showspaces=false,               % show spaces adding particular underscores
  showstringspaces=false,         % underline spaces within strings
  showtabs=false,                 % show tabs within strings adding particular underscores
  frame=single,                   % adds a frame around the code
  rulecolor=\color{black},        % if not set, the frame-color may be changed on line-breaks within not-black text (e.g. commens (green here))
  tabsize=2,                      % sets default tabsize to 2 spaces
  captionpos=b,                   % sets the caption-position to bottom
  breaklines=true,                % sets automatic line breaking
  breakatwhitespace=false,        % sets if automatic breaks should only happen at whitespace
  title=\lstname,                 % show the filename of files included with \lstinputlisting;
                                  % also try caption instead of title
%  keywordstyle=\color{blue},      % keyword style
%  commentstyle=\color{green},   % comment style
%  stringstyle=\color{blue},       % string literal style
  %escapeinside={\%*}{*)},         % if you want to add a comment within your code
  escapeinside={(*@}{@*)},         
  morekeywords={*,...}            % if you want to add more keywords to the set
} 

\title{NetSec Labreport 1}
\author{Kolja Hopfmann \\ Kolja Hopfmann \\Universität Hamburg, Projekt: Netzsicherheit}
\date{\today}

\begin{document}
\pagenumbering{arabic}
\maketitle
\centerline{\rule{1.2\linewidth}{.2pt}}
\begin{linenumbers}
\section{Abeiten mit der Linux-Kommandozeile(bash)}
\subsection{Befehle man und help}
Mit den Befehlen man(Manual) und help(--help flag) lassen sich nützliche Informationen über ein Programm über die Kommandozeile einblenden lassen. "man" öffnet hierbei ein Textfeld und gibt eine detaillierte Dokumentation zu dem Programm aus. "--help" gibt eine kurze Beschreibung und mögliche flags für das Programm aus.
\subsection{Script}
Script ist ein Unix-Programm welches die momentane Shell-Session in einer Datei abspeichert bzw. aufzeichnet. Mit "Script Dateiname.txt" startet die Aufzeichnung und alle eingetippten Befehle landen in Dateiname.txt. Wird keine Datei angegeben wird eine default-Datei verwendet. Dieses tool kann dabei helfen mögliche Lösungen für Aufgaben zu präsentieren.
\section{Benutzerkonten und Verwaltung}
\subsection{test}
\section{Datei- und Rechteverwaltung}
Der Benutzer wurde von user zu labmate gewechselt. \\
Das Verzeichnis /home/labreports wurde angelegt. \\
Im Verzeichnis labreports wurde mit "touch bericht1.txt" eine neue Datei angelegt. Anschließend wurde diese über vim mit dem Text "dies ist ein Test" befüllt.\\
Mit dem Befehl "chmod g+rw bericht1.txt" wurden die Rechte auf der Datei so gesetzt, dass für Eigentümer und Gruppenmitglieder Lese- und Schreibrechte vergeben wurden. Die Gruppe der Datei bericht1.txt ist labortests.\\
Mit "wget http://www.uni-hamburg.de/index.html" wurde der Inhalt der Seite in das Verzeichnis labreports gespeichert.\\
Die Zugreiffberechtigung wurde mit "chmod 0660 labreports" so geändert, dass Dateieigentümer und Gruppe Lese- und Schreibzugriff haben.\\
Ein neues Verzeichnis test wurde unter /opt angelegt. Die Gruppe von test wurde auf user gesetzt, der Owner ist labmate.
Die Rechte von test wurden mit "chmod 0770 test" auf Lese- Schreib- und Executezugriff (rwx) für Owner und Gruppenmitglieder gesetzt.In der Zahl mit der die Rechte gesetzt werden steht die 2. Stelle für die rechte des Owners, die 3. Stelle für die Rechte der Gruppenmitglieder und die 4. Stelle für die Rechte aller anderen. Die Flags r, w und x werden durch jeweils ein Bit dargestellt, wobei $r=100_2=4_{10}$, $w=010_2=2_{10}$ und $x=001_2=1_{10}$ gilt.\\
Mit "cp index.html test" wurde die Datei index.html von labreports nach test kopiert.\\
Mit "groupadd specialrights" wurde eine neue Gruppe angelegt. Dieser Gruppe wurden user und labmate hinzugefügt, die Gruppe von index.html wurde auf specialrights geändert und der Owner auf labmate gesetzt. Dies wäre im Nachhinein auch einfacher gegangen, indem die Gruppe von indes.html auf user gesetzt wird. Die Rechte von index.html wurden mit "chmod 0640 index.html" so gesetzt, dass der Owner (labmate) die Lese- und Schreibrecht hat und die Gruppenmitglieder (hier user) Leseberechtigung haben.\\
Mit exit wurde der user labmate ausgeloggt und anschließend der user user angemeldet.\\
Mit dem Befehl "cat index.html" lies sich die Datei erfolgreich auslesen.\\
Die Datei wurde mit "vim index.html" erfolgreich geöffnet. Ein Abspeichern war nicht möglich, da die Datei read-only war.\\
Unter dem Benutzer user wurde die Datei index.html kopiert. Ein Bearbeiten und Abspeichern der Datei war nun möglich, da user der Owner der Kopie war und somit rw Berechtigung hatte.\\
Mit "rm index.html" wurde als user erfolgreich die Datei gelöscht, von der labmate der Owner war. Dies war möglich, da für rm die Schreibberechtigung auf dem übergeordnetem Verzeichnis nötig ist, nicht die Berechtigung auf der Datei selber.
\section{Administration und Akualisierung}
\section{Prozesse und Prozessverwaltung}
\section{VMware-Tools}
\section{Bedienung von VMware}
Die vm wurde gestartet und user angemeldet und mit top die Prozessübersicht geöffnet.\\
Die Vm wurde Pausiert. Mit Resume wurde die VM wieder gestartet, hierbei wurde sie am pausierten Zustand fortgesetzt.\\
Der Wechsel zur Full-Screen ansicht und wieder zurückwar erfolgreich.\\
Top wurde ausgeführt und ein Snapshot mit dem namen top erzeugt.\\
Top wurde beendet, ein Browser-Fenster geöffnet und hiervon ebenfalls ein Snapshot erzeugt.\\
Beim Wiederherstellen des top Snapshots wurde die VM an dem gespeicherten Punkt (top geöffnet) fortgesetzt.\\
Die Anwendung Calculator wurde gestartet und ein 3. Snapshot erstellt. Die Snapshots sind chronologisch angeordnet.\\
Die Snapshots wurden gelöscht und die VM beendet.
\end{linenumbers}
\end{document}