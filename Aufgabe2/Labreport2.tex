\documentclass[12pt]{article}


\usepackage{amssymb}
\usepackage{amsmath}
\usepackage[utf8]{inputenc}
\usepackage[ngerman]{babel}
\usepackage{lineno}
\usepackage{listings}
\usepackage[T1]{fontenc}
\usepackage[utf8]{inputenc}
\usepackage{lmodern}
\usepackage{eurosym}
\usepackage{listings}
\usepackage{microtype}
\usepackage{units}
\usepackage{color}
\usepackage{xcolor}
\usepackage{graphicx}
\usepackage{subfigure}
\usepackage{import}
\usepackage{url}
\usepackage{amsthm}
\theoremstyle{plain}

\lstset
{ %
  language=R,                     % the language of the code
  basicstyle=\footnotesize\ttfamily,       % the size of the fonts that are used for the code
  numbers=left,                   % where to put the line-numbers
  numberstyle=\tiny\color{gray},  % the style that is used for the line-numbers
  stepnumber=1,                   % the step between two line-numbers. If it's 1, each line
                                  % will be numbered
  numbersep=5pt,                  % how far the line-numbers are from the code
  backgroundcolor=\color{white},  % choose the background color. You must add \usepackage{color}
  showspaces=false,               % show spaces adding particular underscores
  showstringspaces=false,         % underline spaces within strings
  showtabs=false,                 % show tabs within strings adding particular underscores
  frame=single,                   % adds a frame around the code
  rulecolor=\color{black},        % if not set, the frame-color may be changed on line-breaks within not-black text (e.g. commens (green here))
  tabsize=2,                      % sets default tabsize to 2 spaces
  captionpos=b,                   % sets the caption-position to bottom
  breaklines=true,                % sets automatic line breaking
  breakatwhitespace=false,        % sets if automatic breaks should only happen at whitespace
  title=\lstname,                 % show the filename of files included with \lstinputlisting;
                                  % also try caption instead of title
%  keywordstyle=\color{blue},      % keyword style
%  commentstyle=\color{green},   % comment style
%  stringstyle=\color{blue},       % string literal style
  %escapeinside={\%*}{*)},         % if you want to add a comment within your code
  escapeinside={(*@}{@*)},         
  morekeywords={*,...}            % if you want to add more keywords to the set
} 

\title{Blatt 2 - Kennwortsicherheit}
\author{Kolja Hopfmann, Jonas Sander}
\date{\today}

\begin{document}
\shorthandoff{"}
\pagenumbering{arabic}
\maketitle
\centerline{\rule{1.2\linewidth}{.2pt}}
\section{Sicherheit lokaler Rechner}
\subsection{}
\subsubsection{}
Die VM wurde mit der grml-CD gebootet und der Festplatteninhalt der eigentlichen VM wurde gemountet.
\begin{lstlisting}
lsblk
mount -r /dev/sda1
\end{lstlisting}
Der Aufbau von $/etc/passwd/$ ist: \\Name:Passwort:UserID:GroupID:Kommentar:Verzeichnis:Shell , wobei hier das Passwort in der $/etc/shadow/$ Datei als Hash ausgelagert ist. \\
Es existieren die User $Georg$ und $webadmin$. Der User $Georg$ hat Sudo-Berechtigung
\subsection{}
\subsubsection{}
Eine kryptographische Hashfunktion ist eine Funktion welche einen beliebigen Eingabeparameter so verändert, dass es praktisch unmöglich ist, mittels Berechnung auf den ursprünglichen Eingabewert zu schließen. Ein Salt ist ein randomisierter Eingabewert, welcher mit dem Passwort konkateniert wird und gemeinsam gehasht wird. Dadurch ist der Hash eines bestimmten Klartextes mit Salt nicht immer der selbe. Die bedeutet das man das Ursprüngliche Passwort nur durch mehrfaches Ausprobieren herausfinden kann.
\subsubsection{}
John wurde installiert und die Manual wurde gelesen:
\begin{lstlisting}
sudo apt install john
man john
\end{lstlisting}
John wurde im Incremental-Mode gestartet:
\begin{lstlisting}
john -incremental /etc/shadow
\end{lstlisting}
Der Versuch war nicht erfolgreich. Grund: Incremental Mode iteriert von Anfang über alle möglichen Passwort-Strings, für einen Zeitraum von 15 Minuten war das Passwort zu lang um es über Incremental zu ermitteln.
\subsubsection{}
Es wurde eine Wordlist runtergeladen,entpackt und anschließend für einen weiteren Versuch mit John The Ripper benutzt.
\begin{lstlisting}
wget http://download.openwall.net/pub/wordlists/all.gz
gunzip all.gz
john --wordlist=all /etc/shadow
\end{lstlisting}
Passwort für $webadmin$: $mockingbird$
\subsection{}
Das Passwort für den User $georg$ war nicht in der Wordlist enthalten. \\ \\
Das VM-Image wurde erneut gemountet, diesmal mit Schreibzugriff. Daraufhin wurde mit $chroot$ eine neue Bash-Session gestartet mit dem VM-Image root als root. Somit konnte man mit Rootrechten das Passwort von $georg$ ändern.
\begin{lstlisting}
mount /dev/sda1
chroot /media/sda1 /bin/bash
passwd georg
exit
\end{lstlisting}
\section{Sichere Speicherung von Kennwörtern}
\subsection{}
1.	Zuerst wurde in das Verzeichnis von rcracki navigiert.
\begin{lstlisting}
cd webadmin/Rainbowtables/rcracki_mt_0.7.0_Linux:x86_64
\end{lstlisting}
Hier wurde dann rcracki auf der Datei mit den Passwörtern ausgeführt.
\begin{lstlisting}
./cracki -l [password txt] [path to rainbow table]
\end{lstlisting}
Hierbei wurden Passwort Nummer 4: ulardi und Passwort Nummer 5: avanti gefunden.
2.	Die Restlichen Passwörter konnten mit der Verwendeten Rainbowtable nicht geknackt werden. Dies ist darauf zurück zu führen, dass die Passwörter nicht als Wörter in der Tabelle enthalten sind. Ein erneuter Versuch mit einer anderen Rainbowtable könnte weitere Passwörter knacken, jedoch ist dies keineswegs sicher.\\
Für das Abspeichern der MD5-Hashes aller alphanumerischen Passwörtern der Länge 1-7 wäre ein Speicher von $\sum_1^7 (62^i \cdot 32 byte) = 1,1 \cdot 10^{14} Byte = 110 TB$ nötig. Dies ist um den Faktor $10^4$ größer als die verwendete Rainbowtable und deutlich zu groß zur Speicherung.
\subsection{}
Unsere Lösung war auf dem Alphabet bestehend aus 0-9 und a-z iterativ alle möglichen Strings durch zu probieren. Hierfür haben wird eine Java Klasse geschrieben, die in einer Schleife von 0 bis $36^7-1$ iteriert. In Jeder Iteration wird ein String zusammengesetzt, dieser mit dem Salt gehasht und das Ergebnis des Hashes mit dem in der Passwort-Datei gefundenem Hash verglichen. Hierbei werden erst alle 1-Stelligen, dann alle 2-Stelligen Passwörter durchlaufen etc. Hiermit wurde das Passwort s1v3s gefunden. Die Datei bruteforce.java befindet sich im Anhang.
\subsection{}
Die erstellte Java Datei Useradmin.java befindet sich im Anhang.
\section{Forensische Wiederherstellung von Kennwörtern}
\subsection{}
Aus den empfohlenen Quellen zu Blatt 2 unter "Zum sicheren Umgang mit Passwörtern im Speicher" geht hervor, dass viele Anwendungen Passwörter als Klartext im Arbeitsspeicher ablegen. Dadurch ist es Möglich das Passwort (für kurze Zeit) selbst nach ausschalten des Systems aus dem Speicher zu extrahieren. Desweiteren Dient eine Swap-Partition als zusätzlicher Arbeitsspeicher. So ist es Möglich aus der Swap-Partition Passwörter zu extrahieren.
\subsection{}
Unter grml wurde der Inhalt der VM gemountet:
\begin{lstlisting} 
mount /dev/sda1
\end{lstlisting}
Unter $/home/user/.bash\_history$ befinden sich die letzten commands welcher der user in der letzten bash-session in die shell eingetippt hat.
\subsection{}
Da der Admin zunächst mit Jedit die Server-File editiert hatte haben wir zunächst in den Jedit config-Files gesucht, unter $home/.jedit$. Da war zu erkennen dass das alte Passwort "Flugentenfederkiel/991199" war, mit dem User "bloguser". \\ \\ Daraufhin wurde versucht mit dem Programm photorec, Teile der Swap-Parition wiederherzustellen:
\begin{lstlisting}
lsblk
photorec /dev/sda5
\end{lstlisting}
Die wiederhergestellten Dateien wurden aufgrund Platzmangel der grml-CD unter $/dev/sda1$ abgelegt. Nun wurde versucht die Dateien nach dem String "PASSWORD" zu untersuchen da dies in der Server-Datei vor dem tatsächlichen Passwort stand.
\begin{lstlisting}
find . -name "*.elf" | xargs grep -E 'PASSWORD'
find . -name "*.txt" | xargs grep -E 'PASSWORD'
find . -name "*.java" | xargs grep -E 'PASSWORD'
grep -a 'PASSWORD' f0402684.elf
grep -a 'PASSWORD' f0853144.elf
grep -a 'PASSWORD' f0932456.elf
grep -a 'PASSWORD' f0052256.elf
\end{lstlisting}
In der Datei $f0853144.elf$ befand sich das Passwort als Klartext: "Kindergeburtstag/119911"
\section{Unsicherer Umgang mit Passwörtern in Java}
Der Passwortmanager verwendet sowohl für Encryption als auch Decryption die selbe Funktion. Dieser wird entweder das Klartextpasswort oder das Verschlüsselte Passwort übergeben. Kommt man in den Besitz der Verschlüsselten Passwörter, kann man sie also einfach über den Passwortmanager entschlüsseln.
\section{Quellen}
\begin{itemize}
\item 0. Empfohlene Quellen
\item \url{https://unix.stackexchange.com/}
\item man command der Kommandozeile
\end{itemize}
\section{Anhang}
\subsection{}
Code des Bruteforcers
\begin{lstlisting}
import javax.xml.bind.DatatypeConverter;
import java.security.MessageDigest;
import java.security.NoSuchAlgorithmException;

public class bruteforce
{

	public static void main(String[] args)
	{
		final String hash = "2b2935865b8a6749b0fd31697b467bd7";
		final String salt = "8kofferradio";
		final String account = "testaccount";
		final int cardinality = 6;
		String result = "";

		final char[] alphabet = {'0', '1', '2', '3', '4', '5', '6', '7', '8', '9',
				'a', 'b', 'c', 'd', 'e', 'f', 'g', 'h', 'i', 'j', 'k', 'l',
				'm', 'n', 'o', 'p', 'q', 'r', 's', 't', 'u', 'v', 'w', 'x', 'y', 'z'};


		for (long i = 0L; i < Math.pow(alphabet.length, cardinality + 1); i++)
		{
			//first arity
			result = alphabet[(int) (i % 36)] + ""; //36^0
			if (i >= Math.pow(36, 1))
			{
				//second arity
				result = alphabet[(int) ((i / 36) - 1) % 36] + result; //36^1
				if (i >= Math.pow(36, 2))
				{
					//third arity
					result = alphabet[(int) ((i / Math.pow(36, 2) - 1) % 36)] + result;
					if (i >= Math.pow(36, 3))
					{
						//fourth arity
						result = alphabet[(int) ((i / Math.pow(36, 3)) - 1) % 36] + result;
						if (i >= Math.pow(36, 4))
						{
							//fifth arity
							result = alphabet[(int) ((i / Math.pow(36, 4)) - 1) % 36] + result;
							if (i >= Math.pow(36, 5))
							{
								//sixth arity
								result = alphabet[(int) ((i / Math.pow(36, 5)) - 1) % 36] + result;
							}
						}
					}
				}
			}

			String saltAndPasswd = salt + result;
			String hashedPasswd = "";
			try
			{
				hashedPasswd = md5Hash(saltAndPasswd);
			}
			catch (NoSuchAlgorithmException ex)
			{
				System.out.println("Error when hashing: ");
				ex.printStackTrace();
			}


			if (hash.equals(hashedPasswd))
			{

				System.out.println("Password found! :");
				System.out.println("User: " + account + " Password: " + result);
				break;
			}
			else
			{
				if(i%100000 == 0)
				{
					System.out.println(i + ". at " + result + "-- Retry");
				}
			}
		}
		System.out.println("############################################### --- *hackervoice* I'M IN --- #################################################");
	}

	private static String md5Hash(String plain) throws NoSuchAlgorithmException
	{
		MessageDigest md5 = MessageDigest.getInstance("MD5");
		md5.update(plain.getBytes());
		byte[] byteHash = md5.digest();
		return DatatypeConverter.printHexBinary(byteHash).toLowerCase();
	}
}

\end{lstlisting}

\subsection{}
Code der Webadmin Klasse
\begin{lstlisting}
package com.hoppix;

import javax.xml.bind.DatatypeConverter;
import java.io.*;
import java.security.MessageDigest;
import java.security.NoSuchAlgorithmException;
import java.util.Random;
import java.util.Scanner;

public class Useradmin implements Useradministration
{

	private static final String fileName = "passwords.txt";
	private static final int hashIterator = 1337;

	public static void main(String[] args)
	{
		String command = args[0];
		String user = args[1];

		Useradmin admin = new Useradmin();

		System.out.println("Password: ");
		Scanner scanner = new Scanner(System.in);
		String password = scanner.next();
		scanner.close();



		switch (command)
		{
			case "addUser":
				admin.addUser(user, password.toCharArray());
				break;

			case "checkUser":
				boolean found = admin.checkUser(user, password.toCharArray());
				if (found)
				{
					System.out.println("User correct");
				}
				break;

			default:
				System.out.println("Wrong input");

		}

	}

	public void addUser(String username, char[] password)
	{
		String passwordString = String.copyValueOf(password);
		String hash = "";

		Random rand = new Random();
		int salt = rand.nextInt(1234567) + 100000;

		try
		{
			hash = md5Hash(salt + passwordString);
			for (int i = 0; i < hashIterator; i++)
			{
				hash = md5Hash(hash);
			}

		}
		catch (NoSuchAlgorithmException e)
		{
			e.printStackTrace();
		}
		finally
		{
			writeLine(fileName, username, salt, hash);
		}


	}

	public boolean checkUser(String username, char[] password)
	{
		String[] content = checkPasswordLine(fileName, username);
		if (content.equals(null))
		{
			System.out.println("User not found!");
			return false;
		}

		String passwordString = String.copyValueOf(password);
		String salt = content[1];
		String hashSave = content[2];

		try
		{
			String hash = md5Hash(salt + passwordString);
			for (int i = 0; i < hashIterator; i++)
			{
				hash = md5Hash(hash);
			}

			return hash.equals(hashSave);
		}
		catch (NoSuchAlgorithmException e)
		{
			System.out.println("Wrong password!");
			e.printStackTrace();
			return false;
		}

	}

	private static String md5Hash(String plain) throws NoSuchAlgorithmException
	{
		MessageDigest md5 = MessageDigest.getInstance("MD5");
		md5.update(plain.getBytes());

		byte[] byteHash = md5.digest();
		return DatatypeConverter.printHexBinary(byteHash).toLowerCase();
	}

	public void writeLine(String fileName, String username, int salt, String saltHashedPassword)
	{
		try
		{
			FileWriter fileWriter = new FileWriter(fileName);
			BufferedWriter bufferedWriter = new BufferedWriter(fileWriter);
			FileReader fileReader = new FileReader(fileName);
			BufferedReader bufferedReader = new BufferedReader(fileReader);

			if (bufferedReader.readLine() == null)
			{
				bufferedWriter.write(username + " " + salt + " " + saltHashedPassword);
				bufferedWriter.newLine();
			}

			bufferedWriter.close();
			bufferedReader.close();
		}
		catch (IOException e)
		{
			e.printStackTrace();
			System.out.println("Schreibfehler");
		}
	}

	public String[] checkPasswordLine(String fileName, String username)
	{
		String line = null;

		try
		{
			FileReader fileReader = new FileReader(fileName);
			BufferedReader bufferedReader = new BufferedReader(fileReader);

			while ((line = bufferedReader.readLine()) != null)
			{
				String[] content = line.split(" ");
				if (username.equals(content[0]))
				{
					return content;
				}
			}
			bufferedReader.close();
			System.out.println("Read " + fileName);
		}
		catch (FileNotFoundException e)
		{
			e.printStackTrace();
			System.out.println("Datei nicht gefunden");
		}
		catch (IOException e)
		{
			e.printStackTrace();
			System.out.println("Lesefehler");
		}
		return null;
	}

}

interface Useradministration
{
	void addUser(String username, char[] password);

	boolean checkUser(String username, char[] password);
}

\end{lstlisting}

\end{document}