\documentclass[12pt]{article}


\usepackage{amssymb}
\usepackage{amsmath}
\usepackage[utf8]{inputenc}
\usepackage[ngerman]{babel}
\usepackage{lineno}
\usepackage{listings}
\usepackage[T1]{fontenc}
\usepackage[utf8]{inputenc}
\usepackage{lmodern}
\usepackage{eurosym}
\usepackage{listings}
\usepackage{microtype}
\usepackage{units}
\usepackage{color}
\usepackage{xcolor}
\usepackage{graphicx}
\usepackage{subfigure}
\usepackage{import}
\usepackage{url}
\usepackage{amsthm}
\theoremstyle{plain}

\lstset
{ %
  language=Java,                     % the language of the code
  basicstyle=\footnotesize\ttfamily,       % the size of the fonts that are used for the code
  numbers=left,                   % where to put the line-numbers
  numberstyle=\tiny\color{gray},  % the style that is used for the line-numbers
  stepnumber=1,                   % the step between two line-numbers. If it's 1, each line
                                  % will be numbered
  numbersep=5pt,                  % how far the line-numbers are from the code
  backgroundcolor=\color{white},  % choose the background color. You must add \usepackage{color}
  showspaces=false,               % show spaces adding particular underscores
  showstringspaces=false,         % underline spaces within strings
  showtabs=false,                 % show tabs within strings adding particular underscores
  frame=single,                   % adds a frame around the code
  rulecolor=\color{black},        % if not set, the frame-color may be changed on line-breaks within not-black text (e.g. commens (green here))
  tabsize=2,                      % sets default tabsize to 2 spaces
  captionpos=b,                   % sets the caption-position to bottom
  breaklines=true,                % sets automatic line breaking
  breakatwhitespace=false,        % sets if automatic breaks should only happen at whitespace
  title=\lstname,                 % show the filename of files included with \lstinputlisting;
                                  % also try caption instead of title
%  keywordstyle=\color{blue},      % keyword style
%  commentstyle=\color{green},   % comment style
%  stringstyle=\color{blue},       % string literal style
  %escapeinside={\%*}{*)},         % if you want to add a comment within your code
  escapeinside={(*@}{@*)},         
  morekeywords={*,...}            % if you want to add more keywords to the set
} 

\title{default title}
\author{default author \\facculty}
\date{00/00/0000}

\begin{document}
\shorthandoff{"}
\pagenumbering{arabic}
\maketitle
\centerline{\rule{1.2\linewidth}{.2pt}}
\section{HTTP}
\subsection{}
\begin{lstlisting}
telnet man7.org 80
GET /linux/man-pages/man8/lsof.8.html HTTP/1.1
Host: www.man7.org
GET /style.css
\end{lstlisting}
\section{SMTP}
\subsection{}
Die Verbindung zum Mailhost wurde mit Telnet hergestellt. Zum Erstellen der Mail wurde die Quelle linuxjournal.com verwendet. Als Mailhost wurde die Adresse des Fachbereichs angegeben und eine Informatik Mailadresse sowohl als Absender als auch Empfänger.\\
Im Mailprogramm werden Absender und Empfänger korrekt angezeigt, im Quelltext der Mail fällt jedoch auf, dass als Host Absender "Unknown Host" angegeben ist.\\
Der Versuch vom Mailhost des Fachbereiches aus eine g-mail-Adresse anzugeben schlug mit \"Cannot find your hostname\" fehl. Der Versuch zum Google Host zu verbinden scheiterte aufgrund eines Timeouts.
\section{DNS-Spoofing}
\subsection{}
$serial={}$ erwartet eine Eingabe von bestimmter Länge, andere Eingaben sind nicht valide und es wird ausgegeben, dass die Länge falsch ist. Durch ausprobieren von Eingaben mit aufsteigender Länge wurde herausgefunden, dass die Eingaben die Länge 8 haben müssen. Bei Falschen Eingaben korrekter Länge wird $SERIAL\_VALID=0$ zurück gegeben, um die Software zu aktivieren wird also $SERIAL\_VALID=1$ als Rückgabe benötigt.
\subsection{}
In der Datei /etc/hosts wird license-server.svslab auf 127.0.0.1 umgeleitet, wo der Spoof-Server gestartet wurde. Dieser gibt bei jeder beliebigen Eingabe $SERIAL\_VALID=1$ zurück. Die Software lässt sich so ohne Kenntnis eines gültigen Keys aktivieren.
\subsection{}
Um sich  gegen Spoofing zu schützen wäre eine Möglichkeit Adressen über die direkte IP aufzulösen statt über Namespaces.\\
Alternativ könnte auf Client und Server eine Hashfunktion definiert sein. Der Client generiert eine Beliebige Zahl und schickt diese an den Server. Der Server schickt die gehashte Zahl an den Client zurück, woraufhin dieser die Ursprüngliche Zahl ebenfalls hasht und mit der vom Server gehashten Zahl vergleicht. Ist dies korrekt, weiß der Client, dass er mit dem echten Server verbunden ist und kann nun einen Key zur Freischaltung der Software schicken. Da der Client nur als class Datei vorliegt wäre es sehr schwer für einen Angreifer die Hashfunktion heraus zu finden.
\section{Brute-Force-Angriff}
\subsection{}
Der Verwendete Bruteforcer bricht nicht nach dem ersten Schlüssel ab, sondern Schreibt alle gefundenen Schlüssel in eine Datei. Basierend auf den am Whiteboard angeschriebenen Schlüsseln wurde die Annahme getroffen, dass alle Schlüssel auf 00 enden. Die Menge der auszuprobierenden Schlüssel wurde so von $10^8$ auf $10^6$ verringert. Code Siehe Anhang
\subsection{}
Schutzmaßnahmen gegen diese Angriffsart wären das Begrenzen wie viele Versuche ein Client hat um falsche Schlüssel zu Schicken oder zu begrenzen wie viele Anfragen mit einem Schlüssel pro Zeiteinheit angenommen werden.
\subsection{}
Der erste gefundene Key ist 03133700. Alle weiteren Schlüssel sind die nächsten vielfachen dieses Schlüssels, insgesamt 31 Stück.
\section{TCP-Chat}
\subsection{}
Ein UDP Socket wurde erstellt, der auf dem Port 9999 mithört und die empfangenen URL Schnipsel ausgibt. Da immer zufällige URL Schnipsel übertragen werden, speichert unser Programm ab, welche Schnipsel seit Start bereits empfangen wurden und gibt nur neue aus. Nach 200 Empfangenen Schnipseln wird der Socket geschlossen, da die Wahrscheinlichkeit einen Schnipsel noch nicht empfangen zu haben ausreichend gering ist.\\
URL1: technetwork/; http://www.oracle.com/; java/; socket-140484.html\\
URL2: /?r=56+basic\%2Fsocket; http://code.google.com/; \\\indent
\quad p/example-of-servlet/source/browse; /trunk/src/javaLanguage/basic\\
Hierraus wurden die URLs zusammengesetzt:\\
\url{http://www.oracle.com/technetwork/java/socket-140484.html}\\
\url{http://code.google.com/p/example-of-servlet/source/browse/trunk/src/javaLanguage/basic/?r=56+basic\%2Fsocket}
\subsection{}
Das Projekt von URL2 wurde Runtergeladen, benötig wurde hiervon der Ordner example-of-servlettrunk\textbackslash src\textbackslash main\textbackslash java\textbackslash javaLanguage\textbackslash basic\textbackslash socket.\\
SocketThrdServer erzeugt die Graphische Oberfläche und startet die listenSocket Methode. Verbindet sich nun ein Client mit dem Server, wird ein neuer Thread mit einem ClientWorker erzeugt. Der ClientWorker wartet nun auf eine Eingabe vom Client. Sobald er eine erhält gibt er diese als Echo an den Client zurück und fügt die Nachricht der Graphischen Textbox hinzu.
\subsection{}
Es wurde eine neue statische Java-Class $MessageManager$ erstellt.
Besitzt die Methoden um neue Clients zu registrieren und wieder zu entfernen, sowie eine Nachricht an alle bestehenden Clients zu senden.
Da die Funktionen des $MessageManager$ zeitgleich von mehreren Threads verwendet werden wird das Schlüsselwort $synchronized$ benötigt um einen sicheren Zugriff zu gewährleisten.
\begin{lstlisting}
class MessageManager {
	private static ArrayList<Socket> connectedClients = new ArrayList<>();

	public synchronized static void sendMessage(String message)
	{
		for(Socket client : connectedClients)
		{
			....			
				PrintWriter out = new PrintWriter(client.getOutputStream(), true);
				out.println(message);
			....
		}
	}
	public synchronized static void registerClient(Socket client)
	{
		connectedClients.add(client);
	}
	public synchronized static void removeClient(Socket client)
	{
			connectedClients.remove(client);
	}
}
\end{lstlisting}
Beim Erstellen einer neuen Client-Verbindung wird der Client nun zu den verbundenen Clients hinzugefügt.
\begin{lstlisting}
ClientWorker w;
	try {
		w = new ClientWorker(server.accept(), textArea);
		MessageManager.registerClient(w.getClient());
		Thread t = new Thread(w);
		t.start();
	}
\end{lstlisting}
In der $run()$ Methode wird nun anstatt des PrintWriters die sendMessage() Funktion verwendet.
\begin{lstlisting}
	line = in.readLine();
	//Send data back to client
	//out.println(line);
	MessageManager.sendMessage(line);
	textArea.append(line + "\n");
\end{lstlisting}
\subsection{}
Für die Authentifizierungsfunktion wurde ein $AuthentificationManager$ erstellt.
Dieser hält eine Liste an zugelassenen Users stellt diese statisch für das System zur Verfügung.
Der ClientWorker prüft nun ob die eingegebenen Daten mit einem User/Passwort-Tupel übereinstimmen und startet nur nach akzeptierter Eingabe die $run()$ Methode.
\begin{lstlisting}
w = new ClientWorker(server.accept(), textArea);
AuthentificationManager.initUserbase();
if(w.verifyUser())
{
	MessageManager.registerClient(w.getClient());
	Thread t = new Thread(w);
	t.start();
}
\end{lstlisting}
\subsection{}
\section{Anhang}
\subsection{Response-Header}
\begin{lstlisting}
Trying 213.131.240.174...
Connected to man7.org.
Escape character is '^]'.
GET /style.css HTTP/1.1
Host: www.man7.org

HTTP/1.1 200 OK
Date: Thu, 03 May 2018 12:30:17 GMT
Server: Apache
Last-Modified: Mon, 30 Apr 2018 19:02:44 GMT
ETag: "b5d-119c-56b157fabbd00"
Accept-Ranges: bytes
Content-Length: 4508
Connection: close
Content-Type: text/css


/*
General color scheme:
#333 Grey for copy text

Links: #336699;
Links visited state: #4488bb;
Other Link states: #07488A;

#666 text grey
#ddd grey
#BBCCD9 blue grey
#4488bb lighter blue
#336699 dk blue
*/

/*
that space that you wanted to get rid of is actually space further up here;
    html, body {
        margin: 0;
        padding: 0;
    }
*/

/* General Styles */
html, body {   
    background-color: #fcfcfc;
    font-family: sans-serif;
    font-size: 100%;
    color: #181818;
}

body {
    margin-top: 0px;
    margin-left: 0px;
    margin-right: 0px;
}

h1, h2, h3, h4, h5, h5 {
    font-family: helvetica, sans-serif;
    font-weight: normal;
    margin-left: 8px;
    margin-right: 8px;
    color: #008000;
    margin-top: 25px;
}

h1 {
/*
    font-size: 18px;
*/
}

h2 {
/*
    font-size: 16px;
*/
}

h3 {
    color: #006000;
    font-style: italic;
}

p {
    margin-left: 8px;
    margin-right: 8px;
    margin-bottom: 0.5em;
}

p {
    max-width: 750px;
}

.review-quote blockquote {
    max-width: 650px;
    margin-left: 80px;
}

.review-quote p {
    border: dashed rgb(250,100,100);
    background-color: rgb(230,255,230);
    border-width: 1px;
    padding: 10px;
    margin-top: 30px;
    max-width: 730px;

}


li {
    max-width: 710px;
}

p.copyright-text {
    margin-top: 0px;
    margin-bottom: 0px;
}

pre {
    margin-left: 8px;
}

li {
    margin-left: 8px;
    margin-right: 8px;
}

ul.list-wide li {
    margin-bottom: 12px;
}

a {
    color: #1030ff;
    text-decoration: none;
}

a:visited {
    color: #4080dd;
}

a:hover, a:focus, a:active {
    color: red;
    background-color: #ffe0e0;
    text-decoration: underline;
}

/* Preamble */


/* End of Interfaces */

a.func {
    color: red;
    text-decoration: none;
}
a.manpage {
    color: red;
    text-decoration: none;
}
a.file {
    color: red;
    text-decoration: none;
}

div.lmargin {
    margin-left: 8px;
}

pre.code {
    background-color: #f4f0f4;
//    font-family: monospace, courier;
    font-size: 110%;
    margin-left: 0px;
    margin-right: 60px;
    padding-top: 5px;
    padding-bottom: 5px;
    padding-left: 8px;
    padding-right: 8px;
    border: 1px solid #AADDAA;
}

div.nav-bar {
    padding-left:8px;
    padding-right:8px;
    padding-top:3px;
    padding-bottom:3px;
    background-color: #e8e8e8;
}

div.footer {
    padding-left:8px;
    padding-right:8px;
    padding-top:3px;
    padding-bottom:3px;
    background-color: #e8e8e8;
}

hr.nav-end {
    height:0px;
    margin-top:0px;
    color: #0000ff;
    border-color:#ffffff;
    width:100%;
}

hr.start-footer, hr.end-footer {
    height:0px;
    margin-top:0px;
    margin-bottom:0px;
    color: #0000ff;
    border-color:#ffffff;
    width:100%;
}

p.indented {
    margin-left: 100px;
}


p.training-text {
    margin-top: 0px;
    margin-left: 0px;
    margin-bottom: 0px;

    font-weight: bold;
    font-size: 15px;
}

strong, span.hl {
    font-weight: bold;
    color: #303030;
}

a.training-link:hover, a.training-link:visited, a.training-link:link,
a.training-link:active {
    color: #008000;
}

a.training-link:hover, a.training-link:active {
    text-decoration: underline;
    background-color: #ffd0d0;
}

/*
a.training-link {
    color: #228822;
}
*/

p.nav-text {
    margin-top: 0px;
    margin-left: 0px;
    margin-bottom: 0px;
}

p.nav-text, p.copyright-text {
    font-size: 15px;
}

.footer p {
    margin-top: 0px;
    margin-bottom: 0px;
}

.cmd {
    font-family: monospace;
    color: darkgreen;
}

.code {
    color: darkgreen;
}

.const {
    font-family: monospace;
    color: #330000;
}

.email {
    font-family: monospace;
    color: #000080;
}

.envvar {
    font-family: monospace;
    color: #330000;
}

.func {
    font-family: serif;
    font-style: italic;
    color: #330000;
}

.man-page {
    font-family: serif;
    font-style: italic;
    color: darkblue;
}

.pathname {
    font-family: monospace;
    color: darkgreen;
}

.url {
    font-family: serif;
    font-style: italic;
    color: #440064;
}

.var {
    font-family: serif;
    font-style: italic;
    color: #330000;
}

.t-del {
    text-decoration: line-through;
    border-bottom: 3px double #ff0000;
    background-color: #ffcccc;
}

.t-add {
    border-bottom: 3px double #00ff00;
    background-color: #ccffcc;
}

hr.errata-expl {
    border: 1px solid;
    color: #006600;
    height: 2px;
    width: 90%;
}

hr.prod-comment {
    border: 0.3px solid;
    color: #4040aa;
    height: 0px;
    width: 90%;
}
Connection closed by foreign host.
\end{lstlisting}
\subsection{fakeServer.js}
\begin{lstlisting}
const net = require("net");


const HOST = '127.0.0.1';
const PORT = 1337;

net.createServer(function(sock) {

	console.log('CONNECTED: ' + sock.remoteAddress +':'+ sock.remotePort);

	// Add a 'data' event handler to this instance of socket
	sock.on('data', function(data) {

		console.log('DATA ' + sock.remoteAddress + ': ' + data);
		// Write the data back to the socket, the client will receive it as data from the server
		sock.write("SERIAL_VALID=1");

	});

	// Add a 'close' event handler to this instance of socket
	sock.on('close', function(data) {
		console.log('CLOSED: ' + sock.remoteAddress +' '+ sock.remotePort);
	});

}).listen(PORT, HOST);

console.log('Server listening on ' + HOST +':'+ PORT);
\end{lstlisting}
\subsection{bruteforceClient.js}
\begin{lstlisting}
const net = require("net");

var client = new net.Socket();
const targetIp = "license-server.svslab";
const port = 1337;
const logfile = "foundkeys.txt" //file needs to exist to work properly.
var keygen = 0;


client.connect(port, targetIp, function() {
  console.log("Connected on: " + targetIp + ":" + port);
  client.write("Kickoff");
  console.log("Kickoff send");
});

client.on("data", function(data) { 
  if(data.includes("SERIAL_VALID=1"))
  {
    var sKey = generateSerialKey(keygen);
    console.log("Key found: " + sKey);
    logFoundKey(sKey);
  }
  
  if(keygen > Math.pow(10, 6))
  {
    console.log("Finished!");
  }
  else
  {
    keygen++;
    client.write("serial=" + generateSerialKey(keygen));  
  } 
});


function generateSerialKey(key)
{
  const digits = key.toString().length();

  while(key.toString().length() < 6)
  {
    key = "0" + key;
  }

  return key + "00";
}

function logFoundKey(str)
{
  if (!str) return;
  fs.appendFile(logfile, str + "\n", function (err)
  {
    if (err)
    {
      return console.log(err);
    }
  });
}
\end{lstlisting}
\subsection{UDP-Client.java}
\begin{lstlisting}
import java.io.*;
import java.net.*;
import java.util.*;

class UDPClient
{
   public static void main(String args[]) throws Exception
   {

      DatagramSocket clientSocket = new DatagramSocket(9999);
      InetAddress IPAddress = InetAddress.getByName("10.1.1.1");
      
      ArrayList<String> alreadyReceived = new ArrayList<>();
      
      int c = 0;
      while(c < 200)
      {
         byte[] receiveData = new byte[2048];
         DatagramPacket receivePacket = new DatagramPacket(receiveData, receiveData.length);
         clientSocket.receive(receivePacket);
         String modifiedSentence = new String(receivePacket.getData());
      
         c++;
         
         if(!alreadyReceived.contains(modifiedSentence))
         {
           System.out.println("FROM SERVER:" + modifiedSentence);
           alreadyReceived.add(modifiedSentence);
         }
      }
      clientSocket.close(); 
   }
}
\end{lstlisting}
\subsection{SocketThrdServer.java}
\begin{lstlisting}
//package javaLanguage.basic.socket;

import java.awt.BorderLayout;
import java.awt.Color;
import java.awt.event.WindowAdapter;
import java.awt.event.WindowEvent;
import java.awt.event.WindowListener;
import java.io.IOException;
import java.net.ServerSocket;

import javax.swing.JFrame;
import javax.swing.JLabel;
import javax.swing.JPanel;
import javax.swing.JTextArea;

import java.awt.Color;
import java.awt.BorderLayout;
import java.awt.event.*;
import javax.swing.*;

import java.io.*;
import java.net.*;
import java.util.*;

class SocketThrdServer extends JFrame{

	   JLabel label = new JLabel("Text received over socket:");
	   JPanel panel;
	   JTextArea textArea = new JTextArea();
	   ServerSocket server = null;



	   SocketThrdServer(){ //Begin Constructor
	     panel = new JPanel();
	     panel.setLayout(new BorderLayout());
	     panel.setBackground(Color.white);
	     getContentPane().add(panel);
	     panel.add("North", label);
	     panel.add("Center", textArea);
	   } //End Constructor

	  public void listenSocket(){
	    try{
	      server = new ServerSocket(4444);
	    } catch (IOException e) {
	      System.out.println("Could not listen on port 4444");
	      System.exit(-1);
	    }
	    while(true){
	      ClientWorker w;
	      try{
	        w = new ClientWorker(server.accept(), textArea);

					AuthentificationManager.initUserbase();
					if(w.verifyUser())
					{
						MessageManager.registerClient(w.getClient());
		        Thread t = new Thread(w);
		        t.start();
					}

	      } catch (IOException e) {
	        System.out.println("Accept failed: 4444");
	        System.exit(-1);
	      }
	    }
	  }

	  protected void finalize(){
	//Objects created in run method are finalized when
	//program terminates and thread exits
	     try{
	        server.close();
	    } catch (IOException e) {
	        System.out.println("Could not close socket");
	        System.exit(-1);
	    }
	  }

	  public static void main(String[] args){
	        SocketThrdServer frame = new SocketThrdServer();
		frame.setTitle("Server Program");
	        WindowListener l = new WindowAdapter() {
	                public void windowClosing(WindowEvent e) {
	                        System.exit(0);
	                }
	        };
	        frame.addWindowListener(l);
	        frame.pack();
	        frame.setVisible(true);
	        frame.listenSocket();
	  }
	}

	class ClientWorker implements Runnable {
	  private Socket client;
	  private JTextArea textArea;

	  ClientWorker(Socket client, JTextArea textArea) {
	   this.client = client;
	   this.textArea = textArea;
	  }

		public Socket getClient()
		{
			return client;
		}

		public boolean verifyUser()
		{
			String username = "!";
			String password = "!";
		 	BufferedReader in = null;
		 	PrintWriter out = null;

			try{
			 in = new BufferedReader(new InputStreamReader(client.getInputStream()));
			 out = new PrintWriter(client.getOutputStream(), true);

			 out.println("USERNAME: ");
			 username = in.readLine();

			 out.println("PASSWORD: ");
			 password = in.readLine();

		 } catch (IOException e) {
			 System.out.println("in or out failed");
			 System.exit(-1);
		 }

		 return checkUserPassword(username, password);
		}

		private boolean checkUserPassword(String user, String password)
		{
			for (Map.Entry<String, String> entry : AuthentificationManager.registeredUsers.entrySet())
			{
				if(user.equals(entry.getKey()) && password.equals(entry.getValue()))
				{
					return true;
				}
			}
			return false;
		}

	  public void run(){
	    String line;
	    BufferedReader in = null;
	    PrintWriter out = null;
	    try{
	      in = new BufferedReader(new InputStreamReader(client.getInputStream()));
	      out = new PrintWriter(client.getOutputStream(), true);
	    } catch (IOException e) {
	      System.out.println("in or out failed");
	      System.exit(-1);
	    }

			out.println("Successfully logged in!");
			
	    while(true){
	      try{
	        line = in.readLine();
	//Send data back to client
	         //out.println(line);
					 MessageManager.sendMessage(line);
	         textArea.append(line + "\n");
	       } catch (IOException e) {
	         System.out.println("Read failed");
	         System.exit(-1);
	       }
	    }
	  }
	}

	class MessageManager {
		private static ArrayList<Socket> connectedClients = new ArrayList<>();

		public synchronized static void sendMessage(String message)
		{
			for(Socket client : connectedClients)
			{
				try
				{
					PrintWriter out = new PrintWriter(client.getOutputStream(), true);
					out.println(message);
				}
				catch(IOException e)
				{
					e.printStackTrace();
					System.out.println("Error occurred, when broadcasting");
				}
			}
		}

		public synchronized static void registerClient(Socket client)
		{
			connectedClients.add(client);
		}

		public synchronized static void removeClient(Socket client)
		{
			connectedClients.remove(client);
		}
	}

	class AuthentificationManager {
		public static Map<String, String> registeredUsers = new HashMap<>();

		public static void initUserbase()
		{
			registeredUsers.put("user1", "123");
			registeredUsers.put("user2", "123");
		}
}
\end{lstlisting}
\section{Quellen}
\begin{thebibliography}{50}
\bibitem[telnet]{Telnet} Telnet, \url{www.chipkin.com/articles/telnet-commands}	
\bibitem [Linuxjournal]{7655} Mails via netcat, \url{www.linuxjournal.com/content/sending-email-netcat}	
\bibitem [Ubuntuusers]{7655} /etc/hosts, \url{www.ubunuusers.com/host}	
\bibitem [Hacksparrow]{7655} tcp, \url{www.hacksparrow.com/tcp-socket-programming-in-node-js.html}	
\bibitem [Systembash]{7655}udp,\url{https://systembash.com/a-simple-java-udp-server-and-udp-client/}	
\end{thebibliography}
\end{document}