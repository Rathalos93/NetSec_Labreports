\documentclass[12pt]{article}


\usepackage{amssymb}
\usepackage{amsmath}
\usepackage[utf8]{inputenc}
%\usepackage[ngerman]{babel}
\usepackage{lineno}
\usepackage{listings}
\usepackage[T1]{fontenc}
\usepackage[utf8]{inputenc}
\usepackage{lmodern}
\usepackage{eurosym}
\usepackage{listings}
\usepackage{microtype}
\usepackage{units}
\usepackage{color}
\usepackage{xcolor}
\usepackage{graphicx}
\usepackage{subfigure}
\usepackage{import}
\usepackage{url}
\usepackage{amsthm}
\theoremstyle{plain}

\lstset
{ %
  language=R,                     % the language of the code
  basicstyle=\footnotesize\ttfamily,       % the size of the fonts that are used for the code
  numbers=left,                   % where to put the line-numbers
  numberstyle=\tiny\color{gray},  % the style that is used for the line-numbers
  stepnumber=1,                   % the step between two line-numbers. If it's 1, each line
                                  % will be numbered
  numbersep=5pt,                  % how far the line-numbers are from the code
  backgroundcolor=\color{white},  % choose the background color. You must add \usepackage{color}
  showspaces=false,               % show spaces adding particular underscores
  showstringspaces=false,         % underline spaces within strings
  showtabs=false,                 % show tabs within strings adding particular underscores
  frame=single,                   % adds a frame around the code
  rulecolor=\color{black},        % if not set, the frame-color may be changed on line-breaks within not-black text (e.g. commens (green here))
  tabsize=2,                      % sets default tabsize to 2 spaces
  captionpos=b,                   % sets the caption-position to bottom
  breaklines=true,                % sets automatic line breaking
  breakatwhitespace=false,        % sets if automatic breaks should only happen at whitespace
  title=\lstname,                 % show the filename of files included with \lstinputlisting;
                                  % also try caption instead of title
%  keywordstyle=\color{blue},      % keyword style
%  commentstyle=\color{green},   % comment style
%  stringstyle=\color{blue},       % string literal style
  %escapeinside={\%*}{*)},         % if you want to add a comment within your code
  escapeinside={(*@}{@*)},         
  morekeywords={*,...}            % if you want to add more keywords to the set
} 

\title{\vspace{-2cm}NetSec: Blatt 4}
\author{Jonas Sander \\ Kolja Hopfmann \\UHH SoSe18}
\date{\today}

\begin{document}
\pagenumbering{arabic}
\maketitle
\centerline{\rule{1.2\linewidth}{.2pt}}
%\shorthandoff{"}
\section{Vertrautmachen mit der Umgebung}
\subsection{}
Die VMs wurden in der angegebenen Reihenfplge gestartet, auf SurfingVM und RouterVM wurde der Benutzer user/user angemeldet.
\subsection{}
Mit dem Befehl /sbin/ifconfig wurden die Daten erhalten:\\
ip-Adresse: 192.68.254.44\\
Gateway: 192.68.254.1\\
DNS-Server: 10.1.1.1\\
\subsection{}
ens33 IP: 172.16.65.139\\
ens36 IP: 192.168.254.1\\
Gateway: 172.16.65.2\\
\subsection{}
Der Ping zu 10.1.1.1 war erfolgreich unter surfen im Internet möglich, der Ping zu 10.1.1.2 war nicht erfolgreich, da der Server nicht verfügbar war.
\section{Sniffing mit tcpdump}
\subsection{}
informieren done
\subsection{}
sudo tcpdump -p -i any host 192.168.254.44 or 10.1.1.1\\
Mit diesem Befehl wurde, alle Nachrichten protokolliert, die von der SurfingVM gesendet oder empfangen werden. 
Die Felder, die in einer DNS-Antwort übertragen werden sehen wie folgt aus:\\
<timestamp> IP <source IP-Adress and port> > <destination IP Adress and port> Flags [<flags>] <sequence from to, ack number| ack> <tcp window> <packet length>\\
Ausgabe siehe Anhang.
\subsection{}
sudo tcpdump -p -i any src 192.168.254.44 or 10.1.1.1 and port 80\\
Ausgabe siehe Anhang.
\subsection{}
sudo tcpdump -p -A -i any src 192.168.254.44 or 10.1.1.1 and port 80\\
Die Pakete sind vermutlich zu grö als dass tcpdump sie komplett abfängt.\\
sudo tcpdump -p -A -nnvvSs 65535 -i any src 192.168.254.44 or 10.1.1.1 and port 80\\
%TODO: snippets aus dem Anhang vergleichen
Ausgaben siehe Anhang
\subsection{}
%TODO: snippets aus dem Anhang einfügen
Der Header ist Base64 Codiert. 
\subsection{}
beendet
\section{Sniffing mit dsniff und urlsnarf}
\subsection{}
\begin{itemize}
\item sudo urlsnarf -i any -v ".*" src 192.168.254.44 or 10.1.1.1 and port 80
\end{itemize}
Urlsnarf wird ausgeführt, sodass auf alle Netwerk-Interfaces gesnifft wird. Source IP-Adresse ist die SurfingVM oder der DNS-Server. So wird der Gesamte HTTP-Verkehr zwischen den beiden aufgezeichnet.
\subsection{}
\begin{itemize}
\item sudo dsniff -i any src 192.168.254.44 or 10.1.1.2
\end{itemize}
Dsniff sucht in der Kommunikation von SurfingVM und Laborserver nach entschlüsselbaren Passwörtern
\section{Sniffing mit Wireshark}
\subsection{}
\subsection{}
\subsection{}
\subsection{}
\subsection{}
\subsection{}
\subsection{}
\subsection{}
\subsection{}
\section{ARP-Spoofing}
\subsection{}
\subsection{}
\subsection{}
\subsection{}
\subsection{}
\subsection{}
\subsection{}
\section{Scanning mit nmap}
\subsection{}
\subsection{}
\subsection{}
\subsection{}
\subsection{}
\section{OpenVAS}
\subsection{}
\subsection{}
\subsection{}
\subsection{}
\subsection{}
\subsection{}
l
\newline
\section{Quellen}
\begin{thebibliography}{50}
\bibitem  [Cyberciti], \url{www.cyberciti.biz/faq/how-to-find-out-default-gateway-in-ubuntu/}
\bibitem [Stackexchange] , \url{unix.stackexchange.com}
\bibitem [Tcpdump], \url{packetpushers.net/masterclass-tcpdump-interpreting-output/}
\bibitem [Wireshark], \url{wiki.wireshark.org/DisplayFilters}
\bibitem [Nmap 1], \url{ubuntuusers.de/nmap}
\bibitem [Nmap 2], \url{garron.me/en/go2linux/wich-service-or-program-listenung-port.html}
\bibitem [OpenVAS], \url{openvas.org/setup-and-start.de}
\bibitem [ARP Spoof], \url{https://en.wikipedia.org/wiki/ARP_spoofing}
\bibitem [DNS], \url{https://en.wikipedia.org/wiki/Name_server#Caching_name_server}
\bibitem [DNS 2], \url{https://en.wikipedia.org/wiki/Multicast_DNS}
\bibitem [POP], \url{https://de.wikipedia.org/wiki/Post_Office_Protocol}
\bibitem [IMAP], \url{https://de.wikipedia.org/wiki/Internet_Message_Access_Protocol}
\bibitem [SMTP], \url{https://de.wikipedia.org/wiki/Simple_Mail_Transfer_Protocol}
\bibitem [Nmap 3], \url{https://de.wikipedia.org/wiki/Nmap}	
\end{thebibliography}
\section{Anhang}
\end{document}
