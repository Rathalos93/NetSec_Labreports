\documentclass[12pt]{article}


\usepackage{amssymb}
\usepackage{amsmath}
\usepackage[utf8]{inputenc}
%\usepackage[ngerman]{babel}
\usepackage{lineno}
\usepackage{listings}
\usepackage[T1]{fontenc}
\usepackage[utf8]{inputenc}
\usepackage{lmodern}
\usepackage{eurosym}
\usepackage{listings}
\usepackage{microtype}
\usepackage{units}
\usepackage{color}
\usepackage{xcolor}
\usepackage{graphicx}
\usepackage{subfigure}
\usepackage{import}
\usepackage{url}
\usepackage{amsthm}
\theoremstyle{plain}

\lstset
{ %
  language=R,                     % the language of the code
  basicstyle=\footnotesize\ttfamily,       % the size of the fonts that are used for the code
  numbers=left,                   % where to put the line-numbers
  numberstyle=\tiny\color{gray},  % the style that is used for the line-numbers
  stepnumber=1,                   % the step between two line-numbers. If it's 1, each line
                                  % will be numbered
  numbersep=5pt,                  % how far the line-numbers are from the code
  backgroundcolor=\color{white},  % choose the background color. You must add \usepackage{color}
  showspaces=false,               % show spaces adding particular underscores
  showstringspaces=false,         % underline spaces within strings
  showtabs=false,                 % show tabs within strings adding particular underscores
  frame=single,                   % adds a frame around the code
  rulecolor=\color{black},        % if not set, the frame-color may be changed on line-breaks within not-black text (e.g. commens (green here))
  tabsize=2,                      % sets default tabsize to 2 spaces
  captionpos=b,                   % sets the caption-position to bottom
  breaklines=true,                % sets automatic line breaking
  breakatwhitespace=false,        % sets if automatic breaks should only happen at whitespace
  title=\lstname,                 % show the filename of files included with \lstinputlisting;
                                  % also try caption instead of title
%  keywordstyle=\color{blue},      % keyword style
%  commentstyle=\color{green},   % comment style
%  stringstyle=\color{blue},       % string literal style
  %escapeinside={\%*}{*)},         % if you want to add a comment within your code
  escapeinside={(*@}{@*)},         
  morekeywords={*,...}            % if you want to add more keywords to the set
} 

\title{\vspace{-2cm}NetSec Blatt5}
\author{Jonas Sander \\ Kolja Hopfmann}
\date{\today}

\begin{document}
\pagenumbering{arabic}
\maketitle
\centerline{\rule{1.2\linewidth}{.2pt}}
%\shorthandoff{"}
\section{Netzwerkeinstellungen}
ClientVM: \\
Ip-Adress: 192.168.254.44\\
Gateway(RouterVM): 192.168.254.2 \\
DNS-Server: 10.1.1.1\\\\
RouterVM: \\
ens36(virt. Netz): \\
Ip-Adress: 192.168.254.2\\\\
ens33(labor Netz): \\
Ip-Adress: 172.16.65.139 \\\\
ServerVM: \\
IP-Adress: 172.16.65.144

\section{Absichern eines Rechners mit iptables}
check firewalregeln mit: iptables -L\\
regeln löschen: iptables -F\\\\
Regeln:\\
sudo iptables -t filter -I INPUT -p tcp --sport 22 -j ACCEPT // <<SSH \\
sudo iptables -t filter -I INPUT -p icmp -j ACCEPT // <<Ping \\
sudo iptables -t filter -I INPUT -p tcp --sport 80 -j ACCEPT // >>HTTP \\
sudo iptables -t filter -I INPUT -p udp --sport 53 -j ACCEPT //\\
sudo iptables -t filter -I INPUT -p tcp --sport 53 -j ACCEPT //\\
sudo iptables -t filter -I INPUT -p tcp --sport 443 -j ACCEPT // >>HTTPS\\
sudo iptables -t filter -I OUTPUT -p tcp --dport 80 -j ACCEPT // >>HTTP \\
sudo iptables -t filter -I OUTPUT -p udp --dport 53 -j ACCEPT //\\
sudo iptables -t filter -I OUTPUT -p tcp --dport 53 -j ACCEPT //\\
sudo iptables -t filter -I OUTPUT -p tcp --dport 443 -j ACCEPT // >>HTTPS\\
sudo iptables -t filter -I OUTPUT -P icmp -j ACCEPT // >>Ping \\
sudo iptables -t filter -A INPUT -j REJECT // Rest incoming blockieren \\
sudo iptables -t filter -A OUTPUT -j REJECT // Rest outcoming blockieren \\
sudo iptables -t filter -A FOWARD -j REJECT // Rest Foward blockieren \\
\\
Test der Regeln: \\\\
Client -> Router mit SSH funktioniert.\\

\section{Quellen}
\begin{thebibliography}{50}
\bibitem  [Cyberciti], \url{www.cyberciti.biz/faq/how-to-find-out-default-gateway-in-ubuntu/}
\bibitem [Stackexchange] , \url{unix.stackexchange.com}
\end{thebibliography}
\end{document}