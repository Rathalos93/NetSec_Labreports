\documentclass[12pt]{article}


\usepackage{amssymb}
\usepackage{amsmath}
\usepackage[utf8]{inputenc}
%\usepackage[ngerman]{babel}
\usepackage{lineno}
\usepackage{listings}
\usepackage[T1]{fontenc}
\usepackage[utf8]{inputenc}
\usepackage{lmodern}
\usepackage{eurosym}
\usepackage{listings}
\usepackage{microtype}
\usepackage{units}
\usepackage{color}
\usepackage{xcolor}
\usepackage{graphicx}
\usepackage{subfigure}
\usepackage{import}
\usepackage{url}
\usepackage{amsthm}
\theoremstyle{plain}

\lstset
{ %
  language=R,                     % the language of the code
  basicstyle=\footnotesize\ttfamily,       % the size of the fonts that are used for the code
  numbers=left,                   % where to put the line-numbers
  numberstyle=\tiny\color{gray},  % the style that is used for the line-numbers
  stepnumber=1,                   % the step between two line-numbers. If it's 1, each line
                                  % will be numbered
  numbersep=5pt,                  % how far the line-numbers are from the code
  backgroundcolor=\color{white},  % choose the background color. You must add \usepackage{color}
  showspaces=false,               % show spaces adding particular underscores
  showstringspaces=false,         % underline spaces within strings
  showtabs=false,                 % show tabs within strings adding particular underscores
  frame=single,                   % adds a frame around the code
  rulecolor=\color{black},        % if not set, the frame-color may be changed on line-breaks within not-black text (e.g. commens (green here))
  tabsize=2,                      % sets default tabsize to 2 spaces
  captionpos=b,                   % sets the caption-position to bottom
  breaklines=true,                % sets automatic line breaking
  breakatwhitespace=false,        % sets if automatic breaks should only happen at whitespace
  title=\lstname,                 % show the filename of files included with \lstinputlisting;
                                  % also try caption instead of title
%  keywordstyle=\color{blue},      % keyword style
%  commentstyle=\color{green},   % comment style
%  stringstyle=\color{blue},       % string literal style
  %escapeinside={\%*}{*)},         % if you want to add a comment within your code
  escapeinside={(*@}{@*)},         
  morekeywords={*,...}            % if you want to add more keywords to the set
} 

\title{\vspace{-2cm}NetSec Blatt5}
\author{Jonas Sander \\ Kolja Hopfmann}
\date{\today}

\begin{document}
\pagenumbering{arabic}
\maketitle
\centerline{\rule{1.2\linewidth}{.2pt}}
%\shorthandoff{"}
\section{Netzwerkeinstellungen}
ClientVM: \\
Ip-Adress: 192.168.254.44\\
Gateway(RouterVM): 192.168.254.2 \\
DNS-Server: 10.1.1.1\\\\
RouterVM: \\
ens36(virt. Netz): \\
Ip-Adress: 192.168.254.2\\\\
ens33(labor Netz): \\
Ip-Adress: 172.16.65.139 \\\\
ServerVM: \\
IP-Adress: 172.16.65.144

\section{Absichern eines Rechners mit iptables}
check firewalregeln mit: iptables -L\\
regeln löschen: iptables -F\\\\
Regeln:\\
sudo iptables -t filter -I INPUT -p tcp --dport 22 -j ACCEPT // <<SSH \\
sudo iptables -t filter -I INPUT -p icmp -j ACCEPT // <<Ping \\
sudo iptables -t filter -I INPUT -p tcp --sport 80 -j ACCEPT // >>HTTP \\
sudo iptables -t filter -I INPUT -p tcp --sport 443 -j ACCEPT // >>HTTPS\\
sudo iptables -t filter -I INPUT -p udp --sport 53 -j ACCEPT //\\
sudo iptables -t filter -I INPUT -p tcp --sport 53 -j ACCEPT //\\
\\
sudo iptables -t filter -I OUTPUT -p tcp --dport 80 -j ACCEPT // >>HTTP \\
sudo iptables -t filter -I OUTPUT -p tcp --dport 443 -j ACCEPT // >>HTTPS\\
sudo iptables -t filter -I OUTPUT -p udp --dport 53 -j ACCEPT //\\
sudo iptables -t filter -I OUTPUT -p tcp --dport 53 -j ACCEPT //\\
sudo iptables -t filter -I OUTPUT -P icmp -j ACCEPT // >>Ping \\
sudo iptables -t filter -I INPUT -p tcp --sport 22 -j ACCEPT // <<SSH \\
\\
sudo iptables -t filter -A INPUT -j REJECT // Rest incoming blockieren \\
sudo iptables -t filter -A OUTPUT -j REJECT // Rest outcoming blockieren \\
sudo iptables -t filter -A FOWARD -j REJECT // Rest Foward blockieren \\
\\
Test der Regeln: \\\\
Client -> Router mit SSH funktioniert.\\
Router -> Client mit SSH funktioniert nicht\, wie gewollt\
Surfen auf der ClientVM ist möglich\\
\\
Seerverdienst mit netcat -l -v -p 5555\\
Verbindungsaufbau nicht erfolgreich wie zu erwarten, da wir eingehende TCP verbindungen Rejecten, connection Refused.\\
Verbindungsaufbau nicht erfolgreich, keine Reaktion, da die eigehenden Pakete gedroppt werden, Timeout.\\
\\
Dynamische Regeln:\\
sudo iptables -t filter -A OUTPUT -m state --state ESTABLISHED,RELATED -j ACCEPT\\
NEW regeln siehe oben\\
INPUT Analog\\
Tests Funktionieren\\
Dynamisch  Regeln in Praxis bevorzugt:\\
Die Meisten Pakete werden über bereits bestehende Verbindungen empfangen/gesendet. Diese Werden mit der ersten Regel ohne weitere Prüfung Akzeptiert, was sehr viel Zeit spart.
\section{Absichern eines Netzwerkes}
Wenn eine ClientVM über die RouterVM im Netz surfen will, so wird die IP der ClientVM durch die der RouterVM Maskiert. Es ist nicht erkennbar von welcher IP des Virtuellen Netzes die Anfrage kommt, nur, dass süe über die RouterVM lief.\\
Ping Client->Server geht wie es sollte\\
Ping Server->Client geht nicht, da die ClientVM maskiert ist, Timeout\\
Die ServerVM kann die ClientVM nicht finden, da sie ihre IP nur im lokalem Netzwerk hat, wo sich die ServerVM nicht befindet. Nach Außenhin ist nur die Maskierte IP bekannt.\\
\\
ACCEPT, all, anywhere, anywhere, state ESTABLISHED,RELEASED
ACCEPT, udp, anywhere, anywhere, udp spt:Domain state NEW\\
ACCEPT, udp, anywhere, anywhere, udp dpt:Domain state NEW\\
ACCEPT, tcp, anywhere, anywhere, tcp spt:Domain state NEW\\
ACCEPT, tcp, anywhere, anywhere, tcp dpt:Domain state NEW\\
REJECT, all, anywhere, 10.0.0.0/8, reject-with icmp-port-unreachable\\
REJECT, all, anywhere, 172.16.65.144, reject-with icmp-port-unreachable\\
ACCEPT, tcp, anywhere, anywhere, tcp dpt:http state NEW\\
ACCEPT, tcp, anywhere, anywhere, tcp spt:http state NEW\\
ACCEPT, tcp, anywhere, anywhere, tcp dpt:https state NEW\\
ACCEPT, tcp, anywhere, anywhere, tcp spt:https state NEW\\
RECEJT, all, anywhere, anywhere, reject-with icmp-port-unreachable\\
=> surfen ist möglich, andere verbindungen können nciht hergestellt werden\\
\\
Eine Regel hinzugefügt an position 2:\\
ACCEPT, tcp, anywhere, 172.16.65.144 state NEW tcp dpt:ssh\\
=> ssh verbindung von Client zu ServerVM jetzt möglich\\
\\
Eine Regel in die NAT table eingefügt:\\
sudo iptables -t nat -A PREROUTING -p tcp -i ens33 --dport 5022 -j DNAT --to-destination 192.168.254.44:22\\
ssh-Verbindung zur ClientVM von Server aus möglich.\\
\\
sudo ifconfig ens33:1 172.16.65.137 netmask 255.255.255.0\\
sudo iptables -t nat -A PREROUTING -d 172.16.65.130 -i ens33 -j DNAT --to-destination 192.168.254.44
ssh-verbindung kann mit 172.16.65.137 hergestellt werden.\\
\\
\section{SSH-TUNNEL}
ACCEPT, all, anywhere, anywhere, state ESTABLISHED,RELEASED\\
ACCEPT, udp, anywhere, anywhere, udp dpt:Domain state NEW\\
ACCEPT, tcp, anywhere, anywhere, tcp dpt:Domain state NEW\\
ACCEPT, tcp, anywhere, 172.16.65.144, state NEW tcp dpt:ssh\\
ACCEPT, tcp, 172.16.65.144, anywhere, state NEW tcp dpt:ssh\\
RECEJT, all, anywhere, anywhere, reject-with icmp-port-unreachable\\
ssh Verbindung kann hergestellt werden.\\
\\
sudo ssh -L 50000:172.16.65.144:80 user@172.16.65.144\\
Verbindung wird erfolgreich aufgebaut, in wireshark sind die TCP pakete des handshakes zu sehen, die restlichen übertragenen Pakete sind nur ssh verschlüsselt zu sehen, http kann nicht erkannt werden.\\
\\
Local Forwarding forwarded nur statisch auf einen Host, surfen ist deshalb auf diese Weise nicht möglich.\\
\\
ssh -D 50000 172.16.65.144\\
Browser Anpassungen:\\
Advanced settings => Manual proxy confiuration: 127.0.0.1, port 50000\\
Check Prox DNS when SOCK5, check use proxy for all protocols\\
Set SocksHost to 27.0.0.1\\
Surfin werkz!\\
\\
in der ClientVM $/edt/ssh/sshd_config$ GatewayPorts auf yes setzen\\
netcat -l -p 5555\\
ssh -R 1337:127.0.0.1:5555 user@172.16.65.144\\
netcat 127.0.0.1 1337\\
\section{OpenVPN}
1.\\
sudo iptables -A FORWARD -i ens36 -o ens33 -m state --state NEW -j ACCEPT\\
sudo iptables -I FORWARD -m state --state RELATED,ESTABLISHED -j ACCEPT\\
sudo iptables -A FORWARD -j REJECT\\
2.\\
openvpn --genkey --secret static.key\\
Key wurde per CopyPaste der static.key datei auf beide Maschienen gebracht.\\
server.conf angepasst wie in openvpn.net beschrieben:\\
dev tun\\
ifconfig 10.42.0.1 10.42.0.2\\
secret static.key\\
3.\\
sudo iptables -A INPUT -m state --stateNEW -p udp --dport 1194 -j ACCEPT\\
sudo iptables -A INPUT -m state --stateNEW -p tcp --dport 1194 -j ACCEPT\\
sudo iptables -A INPUT -j REJECT\\
sudo iptables -I INPUT -m state --state RELATED, ESTABLISHED -j ACCEPT\\
sudo iptables -A OUTPUT -m state --state RELATED, ESTABLISHED -j ACCEPT\\
sudo iptables -A OUTPUT -j REJECT\\
4.\\
auf der ClientVM die remote.conf anpassen:\\
remote 172.16.65.144\\
dev tun\\
ifconfig 10.42.0.2 10.42.0.1\\
secret static.key\\
5.\\
sudo openvpn remote.conf\\
"Peer Connection Initialized with [AF\_INET]172.16.65.144:1194"\\
"Initialization Sequence Completed"\\
6.\\
- Die Nachrichten werden von beiden Seiten über das Interface tun0 übertragen.\\
- Der Zugriff gelingt nicht\\
- sudo iptables -I INPUT 2 -m state --state NEW -i tun0 -j ACCEPT\\
=> verbindung zum Webserver jetzt möglich\\
- ssh verbindung über tun0 interface möglich\\
\section{HTTP-Tunnel}
\section{Quellen}
\begin{thebibliography}{50}
\bibitem  [Cyberciti], \url{www.cyberciti.biz/faq/how-to-find-out-default-gateway-in-ubuntu/}
\bibitem [Stackexchange] , \url{www.unix.stackexchange.com}
\bibitem [Pro-Linux], \url{www.pro-linux.de}
\bibitem [Serverfault], \url{www.serverfault.com}
\bibitem [computerhope], \url{www.computerhope.com}
\end{thebibliography}
\end{document}