\documentclass[12pt]{article}


\usepackage{amssymb}
\usepackage{amsmath}
\usepackage[utf8]{inputenc}
%\usepackage[ngerman]{babel}
\usepackage{lineno}
\usepackage{listings}
\usepackage[T1]{fontenc}
\usepackage[utf8]{inputenc}
\usepackage{lmodern}
\usepackage{eurosym}
\usepackage{listings}
\usepackage{microtype}
\usepackage{units}
\usepackage{color}
\usepackage{xcolor}
\usepackage{graphicx}
\usepackage{subfigure}
\usepackage{import}
\usepackage{url}
\usepackage{amsthm}
\theoremstyle{plain}


\lstset
{ %
  language=R,                     % the language of the code
  basicstyle=\footnotesize\ttfamily,       % the size of the fonts that are used for the code
  numbers=left,                   % where to put the line-numbers
  numberstyle=\tiny\color{gray},  % the style that is used for the line-numbers
  stepnumber=1,                   % the step between two line-numbers. If it's 1, each line
                                  % will be numbered
  numbersep=5pt,                  % how far the line-numbers are from the code
  backgroundcolor=\color{white},  % choose the background color. You must add \usepackage{color}
  showspaces=false,               % show spaces adding particular underscores
  showstringspaces=false,         % underline spaces within strings
  showtabs=false,                 % show tabs within strings adding particular underscores
  frame=single,                   % adds a frame around the code
  rulecolor=\color{black},        % if not set, the frame-color may be changed on line-breaks within not-black text (e.g. commens (green here))
  tabsize=2,                      % sets default tabsize to 2 spaces
  captionpos=b,                   % sets the caption-position to bottom
  breaklines=true,                % sets automatic line breaking
  breakatwhitespace=false,        % sets if automatic breaks should only happen at whitespace
  title=\lstname,                 % show the filename of files included with \lstinputlisting;
                                  % also try caption instead of title
%  keywordstyle=\color{blue},      % keyword style
%  commentstyle=\color{green},   % comment style
%  stringstyle=\color{blue},       % string literal style
  %escapeinside={\%*}{*)},         % if you want to add a comment within your code
  escapeinside={(*@}{@*)},         
  morekeywords={*,...}            % if you want to add more keywords to the set
} 

\title{\vspace{-2cm}NetSec Blatt 6}
\author{Jonas Sander \\ Kolja Hopfmann \\UHH}
\date{\today}

\begin{document}
\pagenumbering{arabic}
\maketitle
\centerline{\rule{1.2\linewidth}{.2pt}}
%\shorthandoff{"}
\section{Absicherung des TCP-Chats mit SSL}
Die Objekte der Klasse Socket wurden durch Objekte der Klasse SSLSocket und SSLServerSocket ersetzt. Mit dem Programm keytool wurde die Datei $myKey$ mit: 
\begin{verbatim}
keytool -genkey -keystore myKey -keyalg RSA
\end{verbatim} 
erstellt und dort der Schlüssel $skeletonkey$ hinterlegt. In den Java args wurde beim Start von Server und Client die Datei mit den Schlüsseln, sowie das zu verwendende Passwort angegeben.\\
\begin{verbatim}
java -Djavax.net.ssl.keyStore=myKey -Djavax.net.ssl.keyStorePassword=skeletonkey 
 -Djava.protocol.handler.pkgs=com.sun.net.ssl.internal.www.protocol 
 -Djavax.net.debug=ssl SecureServer
\end{verbatim}
\begin{verbatim}
java -Djavax.net.ssl.keyStore=myKey -Djavax.net.ssl.keyStorePassword=skeletonkey 
 SecureClient
\end{verbatim}
Siehe Code im Anhang.
\section{CAs und Webserver-Zertifikate}
\subsection{Fallstudie}
keine Doku nötig
\subsection{OpenSSL selbstsigniertes Zertifikat}
sudo openssl req -x509 -nodes -newkey rsa:2048 -keyout apache-selfsigned.key -out apache-selfsigned.crt\\
für das erzeugen des Zertifikats und des Schlüsseln. Das Zertifikat und der Schlüssel werden dann in der default-ssl.conf eingebunden. Daraufhin wird der Server neugestartet.
\subsubsection{Aufrufen im Browser}
Der Webserver wird nun über https aufgerufen. Das Selbstsignierte Zertifikat wird als vertrauendswürdig eingestuft.
\subsubsection{Nachteil von selbstsignierten Zertifikaten}
Jeder kann ein selbst signiertes Zertifikat erstellen, das macht es noch lange nicht vertrauenswürdig. Aus diesem Grund gibt es neutrale CA's welche Vertrauenswürdigkeit garantieren.
\subsection{HTTPS-Weiterleitung}
Modul aktivieren mit: sudo a2enmod rewrite\\
Erstellen einer .htaccess Datei für das Beschreiben der Regeln (Pfad zur Datei wird in default.conf hinterlegt.) Regeln einfügen:\\
\begin{verbatim}
RewriteEngine On
RewriteCond \%{HTTPS} off
RewriteRule (.*) https://%{HTTP_HOST} %{REQUEST_URI} [R=301,L] 
\end{verbatim}
Server neustarten.
\subsection{sslstrip}
sslstrip und wireshark installiert\\
bei aufruf der Seite über hhtp wurden wir auf https umgeleitet, die TLS-Datenpakete konnten nicht mitgelesen werden (verschlüsslt).\\
sslstrip -p -l 8080\\
http proxy: 127.0.0.1, use for all protocols, no proxy for localhost entfernt\\
Bei aufruf mit http wird nun nicht mehr umgeleitet\\
in der sslstip.log sind Passwort und User der Anmeldung sichtbar.\\
Die Lösung aus 2.3 bietet keine hohe sicherheit, da sie sich mit einfachen Mitteln umgehen lässt.\\
HSTS:\\
HSTS enthält strict-transport-security-header, der dem Client vorgibt, über wie lange Zeit er mit dem Server nur über https kommunizieren darf. Wird hiergegen verstoßen werden alle weiteren Verbindungen vom Server ignoriert. Selbst wenn ein Proxy verwendet wird, gelangt der HSTS header trotzdem an den Browser und dieser muss sich daran halten. Hält sich dieser hieran, ist ein ssl stripping nicht möglich.
\section{Unsichere selbstentwickelte Verschlüsselungsalgorithmen}
\subsection{BaziCrypt}
Der Key lässt sich aus der Verschlüsselten Datei auslesen, da dieser direkt dargestellt wird, wenn 0 werte codiert werden.\\
Key01: e53k9dxe53\\
Key02: 7dmKdn7dmK\\
Key03: Rl9sN3c4Rl\\
Die Dateien wurden mit einem Script entschlüsselt (siehe Anhang).\\
Dieses findet den Key, indem es die 10 letzten Byte der Datei nimmt und XORt die Datei dann mit diesem.\\
Nachricht 1: Hallo Peter, Endlich koennen wir geheim kommunizieren! Bis bald, Max\\
Nachricht 2: Hi Max Super, Sicherheitsbewusstsein ist ja extrem wichtig! Schoene Gruesse, Peter.\\
Nachricht 3: Hi Peter, hast du einen Geheimtipp fuer ein Buch fuer mich? Gruss, Max\\
\subsection{AdvaziCrypt}
Mit Advazicrypt ist das Knacken der Verschlüsselung schwerer, da der key sich nicht mehr direkt aus dem Padding auslesen lässt. Da Jedoch bekannt ist, dass die Gepaddeten Bytes jeweils genau den Wert entsprechend der Anzahl der gepaddeten Bytes enthalten, kann diese Anzahl festgestellt und somit der Schlüssel durch Bitweises XOR dieser Zahl mit dem Verschlüsselten Wert der Gepaddeten Bytes immer noch ermittelt werden. Das Script hierfür befindet sich im Anhang.\\
Key04: qIk4n3oqIk\\
key05: 4i0nsK4i0n\\
key06: 3l337sec3l\\
Nachricht 4: Hi Max, natuerlich: Kryptologie von A. Beutelspacher ist super. Gruss Peter\\
Nachricht 5: Hi Peter, worum geht es in dem Buch? Ciao, Max.\\
Nachricht 6: Hi Max, das ist ein super Buch, das viele Krypto-Themen abdeckt. Gruss Peter\\
\section{EasyAES}
Zuerst werden die möglichen Keys generiert. Dies geschieht über 4 verschachtelte Schleifen, von denen 2 die Position (0-15) der nicht-null Bytes bestimmen und 2 den Inhalt (0-255) dieser. Insgesamt ergeben sich so $(\sum_{i=1}^{16} i^2)\cdot 2^{16}$ unterschiedliche Schlüssel. Über diese Schlüssel wird dann Iteriert und der Klartext mit dem Schlüssel der jeweiligen Iteration Chiffriert und der verschlüsselte Text Dechiffriert. Die beiden hierdurch erhaltenen Werte werden in mit dem verwendeten Schlüssel in die HashMaps middleA und middleB gespeichert. Sobald Jeder Key zur Chiffrierung und Dechiffrierung verwendet wurde, werden alle Werte aus middleA mit allen Werten aus middleB verglichen. Sobald eine Übereinstimmung gefunden wurde, sind die zu diesen Werten gehörenden Keys die zur Verschlüsselung Verwendeten Keys. Zuletzt wird der Klartext mit diesen beiden Keys Chiffriert und das Ergebnis mit dem Verschlüsselten Text verglichen um sicher zu stellen, dass die Keys tatsächlich korrekt sind.\\
Der Code hierfür befindet sich im Anhang, es kommt jedoch zu einer Exception, da die Maps größer als der maximale integer Wert werden.\\
Ein Alternativer Ansatz wäre gar keine Maps zu Speicherung der Encrypteten bzw. Decrypteten zwischenwerte zu verwenden, sondern alle Werte on-the-fly zu generieren und direkt zu verleichen. Hierdurch ist Speicherplatz kein Problem mehr, jedoch müssen alle Werte derart vielfach berechnet werden, dass die Laufzeit unverwendbar schlecht wird.\\
Der Code hierfür befindet sich ebenfalls im Anhang.
\section{Timing-Angriff auf Passwörter}
\section{Quellen}
\begin{thebibliography}{50}
\bibitem [SSLSockets] , \url{https://stackoverflow.com/questions/13874387/create-app-with-sslsocket-java}
\bibitem [Zertifikat mit Apache2], \url{https://www.digitalocean.com/community/tutorials/how-to-create-a-self-signed-ssl
-certificate-for-apache-in-ubuntu-16-04}
\bibitem [Padding] , \url{https://en.wikipedia.org/wiki/Padding_(cryptography)#PKCS7}
\bibitem [Meet-the-Middle] , \url{https://en.wikipedia.org/wiki/Meet-in-the-middle_attack}
\end{thebibliography}
\section{Anhang}
\subsection{Absicherung des TCP-Chats mit SSL}
\subsubsection{ClientWorker}
\begin{lstlisting}
import java.io.BufferedReader;
import java.io.IOException;
import java.io.InputStreamReader;
import java.io.PrintWriter;
import java.net.Socket;
import java.util.Date;
import java.util.Map;

import javax.net.ssl.SSLContext;
import javax.net.ssl.SSLSocket;
import javax.swing.JTextArea;

class ClientWorker implements Runnable
{
	private SSLSocket client;
	private JTextArea textArea;

	ClientWorker(SSLSocket client, JTextArea textArea)
	{
		this.client = client;
		this.textArea = textArea;
	}

	public boolean verifyUser()
	{
		PrintWriter out = null;
		{
			try
			{
				out = new PrintWriter(client.getOutputStream(), true);
			}
			catch (IOException e)
			{
				System.out.println("in or out failed");
				System.exit(-1);
			}
		}
		out.println("Successfully logged in!");
		return true;
	}


	public void run()
	{
		String line;
		BufferedReader in = null;
		PrintWriter out = null;
		try
		{
			in = new BufferedReader(new InputStreamReader(client.getInputStream()));
			out = new PrintWriter(client.getOutputStream(), true);
		}
		catch (IOException e)
		{
			System.out.println("in or out failed");
			System.exit(-1);
		}

		if (this.verifyUser())
		{
			MessageManager.registerClient(client);
			while (true)
			{
				try
				{
					line = in.readLine();
					//Send data back to client
					MessageManager.sendMessage(line);
					textArea.append(line + "\n");
				}
				catch (IOException e)
				{
					e.printStackTrace();
					System.out.println("Read failed");
					System.exit(-1);
				}
			}
		}
	}
}
\end{lstlisting}
\subsection{Unsichere selbstentwickelte Verschlüsselungsalgorithmen}
\subsubsection{BaziDecrypt}
\begin{lstlisting}
const fs = require('fs');

const sFile = "baziCryptfiles/n03.txt.enc" //filename goes here

fs.readFile(sFile, function read(err, data) {
	if (err) {
		throw err;
	}
	console.log(data);
	var sKey = data.slice(data.length -10, data.length);

	console.log("Key found: " +
	Buffer.from(sKey, 'hex').toString('utf8'));

	var clear = decrypt(sKey, data);
	console.log("Klartext: " + clear);

});


function decrypt(sKey, sMessage) {
	var sFullKey = "";

	while(sFullKey.length < sMessage.length) {
		sFullKey += sKey;
	}

	return Buffer.from(xor(sFullKey, sMessage), 'hex').toString('utf8');
}

function xor(a, b) {
	if (!Buffer.isBuffer(a)) a = new Buffer(a)
	if (!Buffer.isBuffer(b)) b = new Buffer(b)
	var res = []
	if (a.length > b.length) {
		for (var i = 0; i < b.length; i++) {
			res.push(a[i] ^ b[i])
		}
	} else {
		for (var i = 0; i < a.length; i++) {
		res.push(a[i] ^ b[i])
		}
	}
	return new Buffer(res);
}
\end{lstlisting}
\subsubsection{MessageManager}
\begin{lstlisting}
import java.io.IOException;
import java.io.PrintWriter;
import java.net.Socket;
import java.util.ArrayList;

public class MessageManager {
private static ArrayList<Socket> connectedClients = new ArrayList<>();

public synchronized static void sendMessage(String message)
{
	for(Socket client : connectedClients)
	{
		try
		{
			PrintWriter out = new PrintWriter(client.getOutputStream(), true);
			out.println(message);
		}
		catch(IOException e)
		{
			e.printStackTrace();
			System.out.println("Error occurred, when broadcasting");
		}
	}
}

	public synchronized static void registerClient(Socket client)
	{
		connectedClients.add(client);
	}

	public synchronized static void removeClient(Socket client)
	{
		connectedClients.remove(client);
	}
}
\end{lstlisting}
\subsubsection{SecureClient}
\begin{lstlisting}
import java.io.DataInputStream;
import java.io.PrintStream;
import java.io.BufferedReader;
import java.io.InputStreamReader;
import java.io.IOException;
import javax.net.ssl.SSLSocket;
import javax.net.ssl.SSLSocketFactory;
import java.net.UnknownHostException;

class SecureClient implements Runnable
{

	// The client socket
	private static SSLSocket clientSocket = null;
	// The output stream
	private static PrintStream os = null;
	// The input stream
	private static BufferedReader is = null;

	private static BufferedReader inputLine = null;
	private static boolean closed = false;

	public static void main(String[] args)
	{
		System.out.println("Main");

		// The default port.
		int portNumber = 4444;
		// The default host.
		String host = "localhost";


		/*
		* Open a socket on a given host and port. Open input and output streams.
		*/
		try
		{
			System.out.println("In try");
			SSLSocketFactory factory = (SSLSocketFactory) SSLSocketFactory.getDefault();
			clientSocket = (SSLSocket) factory.createSocket(host, portNumber);

			inputLine = new BufferedReader(new InputStreamReader(System.in));
			os = new PrintStream(clientSocket.getOutputStream());
			//is = new DataInputStream(clientSocket.getInputStream());
			is = new BufferedReader(new InputStreamReader(clientSocket.getInputStream()));
			System.out.println("All executed");
		}
		catch (UnknownHostException e)
		{
			System.err.println("Don't know about host " + host);
		}
			catch (IOException e)
		{
			e.printStackTrace();
			System.err.println("Couldn't get I/O for the connection to the host "
			+ host);
		}

		/*
		* If everything has been initialized then we want to write some data to the
		* socket we have opened a connection to on the port portNumber.
		*/
		System.out.println(os);
		System.out.println(is);
		System.out.println(clientSocket);
		System.out.println();
		if (clientSocket != null && os != null && is != null)
		{
			System.out.println("Check for socket");
			try
			{

				System.out.println("Starting Thread");
				/* Create a thread to read from the server. */
				new Thread(new SecureClient()).start();
				while (!closed)
				{
					os.println(inputLine.readLine().trim());
				}
				/*
				* Close the output stream, close the input stream, close the socket.
				*/
				os.close();
				is.close();
				clientSocket.close();
			}
			catch (IOException e)
			{
				System.err.println("IOException:  " + e);
			}
		}
	}

	/*
	* Create a thread to read from the server. (non-Javadoc)
	*
	* @see java.lang.Runnable#run()
	*/
	public void run()
	{
		/*
		* Keep on reading from the socket till we receive "Bye" from the
		* server. Once we received that then we want to break.
		*/
		System.out.println("Starting Run");
		String responseLine;
		try
		{
			System.out.println("Starting while");
			while ((responseLine = is.readLine()) != null)
			{
				System.out.println(responseLine);
				//if (responseLine.indexOf("*** Bye") != -1)
				//break;

			}
			System.out.println("Broke out of while");
			closed = true;
		}
		catch (IOException e)
		{
			e.printStackTrace();
			System.err.println("IOException:  " + e);
		}
	}
}
\end{lstlisting}
\subsubsection{SecureServer}
\begin{lstlisting}
//package javaLanguage.basic.socket;

import java.awt.BorderLayout;
import java.awt.Color;
import java.awt.event.WindowAdapter;
import java.awt.event.WindowEvent;
import java.awt.event.WindowListener;
import java.io.IOException;
import java.net.ServerSocket;

import javax.net.ssl.SSLServerSocket;
import javax.net.ssl.SSLServerSocketFactory;
import javax.net.ssl.SSLSocket;
import javax.swing.JFrame;
import javax.swing.JLabel;
import javax.swing.JPanel;
import javax.swing.JTextArea;

class SecureServer extends JFrame
{

	JLabel label = new JLabel("Text received over socket:");
	JPanel panel;
	JTextArea textArea = new JTextArea();
	ServerSocket server = null;

	SecureServer()
	{
		//Begin Constructor
		panel = new JPanel();
		panel.setLayout(new BorderLayout());
		panel.setBackground(Color.white);
		getContentPane().add(panel);
		panel.add("North", label);
		panel.add("Center", textArea);
	} //End Constructor

	public void listenSocket()
	{
	
		try
		{
			SSLServerSocketFactory factory = (SSLServerSocketFactory) SSLServerSocketFactory.getDefault();
			server = (SSLServerSocket) factory.createServerSocket(4444);
		}
		catch (IOException e)
		{
			System.out.println("Could not listen on port 4444");
			System.exit(-1);
		}
		while (true)
		{
			ClientWorker w;
			try
			{
				w = new ClientWorker((SSLSocket) server.accept(), 	textArea);
				Thread t = new Thread(w);
				t.start();
			}
			catch (IOException e)
			{
				System.out.println("Accept failed: 4444");
				System.exit(-1);
			}
		}
	}

	protected void finalize()
	{
		//Objects created in run method are finalized when
		//program terminates and thread exits
		try
		{
			server.close();
		}
		catch (IOException e)
		{
			System.out.println("Could not close socket");
			System.exit(-1);
		}
	}

	public static void main(String[] args)
	{
	System.setProperty("https.protocols", "TLSv1,TLSv1.1,TLSv1.2");
	SecureServer frame = new SecureServer();
	frame.setTitle("Server Program");
	WindowListener l = new WindowAdapter()
	{
		public void windowClosing(WindowEvent e)
		{
			System.exit(0);
		}
	};
	frame.addWindowListener(l);
	frame.pack();
	frame.setVisible(true);
	frame.listenSocket();
	}
}
\end{lstlisting}
\subsubsection{AdvaziDecrypt}
\begin{lstlisting}
const fs = require('fs');

const sFile = "advaziCryptfiles/n06.txt.enc" //filename goes here

fs.readFile(sFile, function read(err, data) {
	if (err) {
		throw err;
	}
	console.log(data);
	var sKey = getAdvaziKey(data);

	console.log("Key found: " +
	Buffer.from(sKey, 'hex').toString('utf8'));

	var clear = decrypt(sKey, data);
	console.log("Klartext: " + clear);

});

function decrypt(sKey, sMessage) {
	var sFullKey = "";

	while(sFullKey.length < sMessage.length) {
		sFullKey += sKey;
	}

	return Buffer.from(xor(sFullKey, sMessage), 'hex').toString('utf8');
}

function xor(a, b) {
	if (!Buffer.isBuffer(a)) a = new Buffer(a)
	if (!Buffer.isBuffer(b)) b = new Buffer(b)
	var res = []
	if (a.length > b.length) {
		for (var i = 0; i < b.length; i++) {
			res.push(a[i] ^ b[i])
		}
	} else {
		for (var i = 0; i < a.length; i++) {
			res.push(a[i] ^ b[i])
		}
	}
	return new Buffer(res);
}


function getAdvaziKey(data) {
	var bPaddedKey = data.slice(data.length -10, data.length);
	var sPaddedKey = Buffer.from(bPaddedKey, 'hex').toString('utf8');
	var sData = Buffer.from(data, 'hex').toString('utf8');
	var sPaddingFill = "";
	var sPaddingFillFull = "";

	var i = 0;
	while(sData.charAt(sData.length - (i+1)) == sPaddedKey.charAt(sPaddedKey.length - ((i%10)+1))) {
		i++;
	}

	sPaddingFill = String.fromCharCode(i);
	console.log("CONVERTED I TO HEXSTRING: " + i + " = " + sPaddingFill);
	
	for(var j = 0; j < 10; j++){
		sPaddingFillFull += sPaddingFill;
	}
	sPaddingFillFull = Buffer.from(new Buffer(sPaddingFillFull), 'hex').toString('utf8');
	console.log(sPaddingFillFull);
	var trueKey = xor(sPaddingFillFull, sPaddedKey);
	console.log(Buffer.from(trueKey, 'hex').toString('utf8'));
	return trueKey;
}

\end{lstlisting}
\subsection{AES-Decryption}
\begin{lstlisting}
import org.apache.commons.codec.binary.Hex;

import java.util.ArrayList;
import java.util.HashMap;
import java.util.HashSet;
import java.util.Map;

/**
* Decrypting easyAES Aufgabe 4.
* Created by hoppix on 05.07.18.
*/
	public class Main
	{
		public static void main(String[] args) throws Exception
		{
			System.out.println("AES Decrypt");
			HashSet<String> keys = generateKeys();
			System.out.println("Done generating Keys");
			Map<String, String> middleA = new HashMap<>();
			Map<String, String> middleB = new HashMap<>();

			String plaintext = "Verschluesselung";
			String ciphertext = "be393d39ca4e18f41fa9d88a9d47a574";
			byte[] bytes = Hex.decodeHex(ciphertext.toCharArray());
			ciphertext = new String(bytes, "UTF-8");

			for (String key : keys) {
				AES.setKey(key);

				AES.encrypt(plaintext);
				middleA.put(AES.getEncryptedString(), key);

				AES.decrypt(ciphertext);
				middleB.put(AES.getDecryptedString(), key );
			}

			System.out.println("Done encrypting");

			for (String sA : middleA.keySet())
			{
				for (String sB : middleB.keySet())
				{
				if(sA.equals(sB))
				{
					ArrayList<String> keyPair = new ArrayList<>();
					String k1 = middleA.get(sA);
					String k2 = middleB.get(sB);

					keyPair.add(k1);
					keyPair.add(k2);

					System.out.println(keyPair.toString());
					System.out.println("Correctness: ");
					System.out.println(checkCorrectness(plaintext, ciphertext, k1, k2));
				}
			}
		}
	}

	private static boolean checkCorrectness(String plaintext, String ciphertext, String key1, String key2)
	{
		AES.setKey(key1);
		AES.encrypt(plaintext);
		String e1 = AES.getEncryptedString();

		AES.setKey(key2);
		AES.encrypt(e1);
		String e2 = AES.getEncryptedString();

		return e2.equals(ciphertext);
	}


	private static HashSet<String> generateKeys()
	{
		HashSet<String> keys = new HashSet<>();

		for (int posByteA = 0; posByteA < 16; posByteA++)
		{
			System.out.println(((double) posByteA / 16) * 100 + "% complete");
			for (int posByteB = posByteA; posByteB < 16; posByteB++)
			{

				for (int valueByteA = 0; valueByteA < 256; valueByteA++)
				{
					for (int valueByteB = 0; valueByteB < 256; valueByteB++)
					{
						byte[] key = new byte[16];
						key[posByteA] = (byte) valueByteA;
						key[posByteB] = (byte) valueByteB;

						//String sKey = Hex.encodeHexString(key);
						String sKey = key + "";
						keys.add(sKey);
					}
				}
			}
		}
		return keys;
	}
}
\end{lstlisting}
\subsubsection{AESDecrypt - no Maps}
\begin{lstlisting}
import org.apache.commons.codec.binary.Hex;

import java.util.ArrayList;
import java.util.HashMap;
import java.util.HashSet;
import java.util.Map;

/**
 * Created by hoppix on 11.07.18.
 */
public class Main
{
		public static void main(String[] args) throws Exception
		{
			System.out.println("AES Decrypt");

			String plaintext = "Verschluesselung";
			String ciphertext = "be393d39ca4e18f41fa9d88a9d47a574";
			byte[] bytes = Hex.decodeHex(ciphertext.toCharArray());
			ciphertext = new String(bytes, "UTF-8");

			for (int posByteA1 = 0; posByteA1 < 16; posByteA1++)
			{
				System.out.println(((double) posByteA1 / 16) * 100 + "% complete");
				for (int posByteB1 = posByteA1; posByteB1 < 16; posByteB1++)
				{
					System.out.println("Iteration " + posByteA1 + "." + posByteB1);
					for (int valueByteA1 = 0; valueByteA1 < 256; valueByteA1++)
					{
						for (int valueByteB1 = 0; valueByteB1 < 256; valueByteB1++)
						{
							for (int posByteA2 = 0; posByteA2 < 16; posByteA2++)
							{
								for (int posByteB2 = posByteA2; posByteB2 < 16; posByteB2++)
								{

									for (int valueByteA2 = 0; valueByteA2 < 256; valueByteA2++)
									{
										for (int valueByteB2 = 0; valueByteB2 < 256; valueByteB2++)
										{
											byte[] key1 = new byte[16];
											byte[] key2 = new byte[16];
											String sKey1 = Hex.encodeHexString(key1);
											String sKey2 = Hex.encodeHexString(key2);

											AES.setKey(sKey1);
											AES.encrypt(plaintext);
											String plainEncrypt = AES.getEncryptedString();
											AES.setKey(sKey2);
											AES.encrypt(ciphertext);
											String cipherDecrypt = AES.getEncryptedString();

											if (plainEncrypt.equals(cipherDecrypt))
											{
												ArrayList<String> keyPair = new ArrayList<>();

												keyPair.add(sKey1);
												keyPair.add(sKey2);

												System.out.println(keyPair.toString());
												System.out.println("Correctness: ");
												System.out.println(checkCorrectness(plaintext, ciphertext, sKey1,
													sKey2));
											}

										}
									}
								}
							}
						}
					}
				}
			}
		}

	private static boolean checkCorrectness(String plaintext, String ciphertext, String key1, String key2)
	{
		AES.setKey(key1);
		AES.encrypt(plaintext);
		String e1 = AES.getEncryptedString();

		AES.setKey(key2);
		AES.encrypt(e1);
		String e2 = AES.getEncryptedString();

		return e2.equals(ciphertext);
	}
}
\end{lstlisting}
\end{document}